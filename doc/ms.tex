\documentclass[a4paper, 12pt]{article}
\usepackage{amsmath}
\usepackage{amssymb}
\usepackage[left=2cm, right=2cm, bottom=3cm, top=2cm]{geometry}
\usepackage{graphicx}
\usepackage[utf8]{inputenc}
\usepackage{microtype}
\usepackage{natbib}

\newcommand{\Cauchy}{\textnormal{Cauchy}}
\newcommand{\Celery}{{\em Celery}}
\newcommand{\Exponential}{\textnormal{Exponential}}
\newcommand{\given}{\,|\,}
\newcommand{\Laplace}{\textnormal{Laplace}}
\newcommand{\location}{\textnormal{location}}
\newcommand{\scale}{\textnormal{scale}}
\newcommand{\Uniform}{\textnormal{Uniform}}


\title{The~\Celery~ model specification}
%\author{Brendon J. Brewer}
\date{}

\begin{document}
\maketitle

%\abstract{\noindent Abstract}

% Need this after the abstract
\setlength{\parindent}{0pt}
\setlength{\parskip}{1em}

Let $\boldsymbol{y} = \{y_1, y_2, ..., y_n\}$ be the vector of measurements
taken at times $\boldsymbol{t} = \{t_1, t_2, ..., t_n\}$. Assume that the
observers have provided ``error bars''
$\boldsymbol{\sigma} = \{\sigma_1, \sigma_2, ..., \sigma_n\}$ along with the
measurements.

\section{Parameters and Hyperparameters}

Let the unknown parameters be $M$, the number of oscillation modes,
$\boldsymbol{T} = \{T_1, T_2, ..., T_M\}$ their periods,
$\boldsymbol{A} = \{A_1, A_2, ..., A_M\}$ their amplitudes,
and $\boldsymbol{Q} = \{Q_1, Q_2, ..., Q_M\}$ their qualities.

The prior for the number of modes $M$ is
\begin{align}
p(M) &\propto \frac{1}{M}
\end{align}
for $M \in \{0, 1, 2, ..., M_{\rm max}\}$. By default, the maximum number of
modes is $M_{\rm max} = 30$.

The priors for these are
\begin{align}
\ln T_i &\sim \Laplace(\location=a_T, \scale=b_T) \\
A_i &\sim \Exponential(\scale=\mu_A) \\
\ln Q_i &\sim \Laplace(\location=a_Q, \scale=b_Q)
\end{align}
where some hyperparameters relating to periods, amplitudes, and qualities
have been introduced. The priors for these hyperparameters are:
\begin{align}
\ln a_T   &\sim \Uniform(10^{-6}t_{\rm range}, t_{\rm range}) \\
\ln b_T   &\sim \Uniform(0, 2)\\
\ln \mu_A &\sim \Cauchy(\location=0, \scale=5)T(-50, 50) \\
\ln a_Q   &\sim \Uniform(0, \ln (1000)) \\
\ln b_Q   &\sim \Uniform(0, 1)
\end{align}
The notation $T(,)$ denotes truncation, as in the BUGS language.

Rough justifications for these choices are as follows:
\begin{itemize}
    \item The amplitudes, all positive, will be scattered around some typical
          value, but we don't know it. The data is likely to be supplied
          in units where amplitudes are probably
          within a few orders of magnitude of unity.
    \item The periods, all positive, will be scattered around some typical
          value we don't know, but the typical period is probably shorter
          (and possibly much shorter) than the time-range of the measurements.
          Periods probably don't vary by much more than an order of magnitude.
    \item The qualities, all positive, will be scattered around some typical
          value that is probably of order tens to hundreds. The mode qualities
          are unlikely to differ from each other by a large margin.
    \item The number of modes is unknown, but for pragmatic computational
          reasons we weakly favour smaller numbers of modes (the prior
          is approximately a discrete log-uniform).
\end{itemize}

%\begin{thebibliography}{999}
%\end{thebibliography}

\end{document}

